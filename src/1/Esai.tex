\documentclass{article}
\usepackage{graphicx} % Required for inserting images
\usepackage{hyperref} % Required for inserting images
\usepackage{listings} % Required for inserting images
\usepackage{color}
\usepackage{minted}

\title{Esai}
\author{Brian Hadian}
\date{February 2026}

\begin{document}

\maketitle

Perkenalkan, Saya Brian A. Hadian dari Teknik Informatika 2023 dan menurut saya, jurusan teknik punya masa depan yang cenderung \textit{stale}, tapi cukup menarik. Jurusan teknik ini biasanya banyak diperlukan karena kapasitas yang dilatih cenderung cukup untuk kebanyakan pekerjaan umum. Selain itu, jurusan teknik juga biasanya punya relasi yang cukup kuat dari alumni-alumninya, tapi juga artinya punya sejarah senioritas yang cukup kental. Hal yang menarik dari kebanyakan jurusan teknik adalah persepsi masyarakat yang cenderung masih salah kaprah soal apa yang dikerjakan sebagai lulusan jurusan teknik. Sebagai contoh, lulusan teknik informatika biasanya dapat dilihat sebagai orang yang bisa memperbaiki printer, atau lulusan teknik elektro dapat dilihat sebagai orang yang dapat membuat handphone dari tangan kosong. Menarik, tapi tidak aplikatif dan beberapa dari lulusan tersebut juga kesulitan untuk membalas percakapan tersebut. 

Berkaitan dengan apa yang sudah dibahas di atas, aku sendiri melihat ada potensi yang bisa dimanfaatkan oleh mahasiswa teknik, yaitu memiliki kemampuan untuk memahami seluruhnya meskipun belum tentu cukup mahir untuk menerapkan aplikasi tersebut. Ini merupakan kesempatan yang dapat diambil dan dapat meningkatkan persona lulusan jurusan teknik di mata masyarakat. Sebagai contoh dari aplikasi tersebut, Saya bergabung dengan tim Aksantara ITB yang menggabungkan beberapa keilmuan, termasuk informatika. Dengan demikian, saya melihat bahwa ada potensi yang dapat dimanfaatkan oleh mahasiswa teknik untuk bisa memahami lebih banyak soal apa yang terjadi di dunia nyata. Karena itu, saya berniat untuk memahami seluruh dasar yang setiap bidang di Aksantara untuk menerapkan dan meningkatkan kualitasnya, meskipun hanya dalam bentuk penelitian. Berdasarkan performa dan apa yang saya alami sejauh ini, aksantara ITB memiliki potensi untuk mengambil bidang lain dapat lebih berpengaruh pada dunia masyarakat umum.

\end{document}
